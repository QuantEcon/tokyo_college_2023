\documentclass[
    xcolor={svgnames,dvipsnames},
    hyperref={colorlinks, citecolor=DeepPink4, linkcolor=DarkRed, urlcolor=DarkBlue}
    ]{beamer}  % for hardcopy add 'trans'

\mode<presentation>
{
  \usetheme{Singapore}
  % or ...
  \setbeamercovered{transparent}
  % or whatever (possibly just delete it)
}

\usefonttheme{professionalfonts}
%\usepackage[english]{babel}
% or whatever
%\usepackage[latin1]{inputenc}
% or whatever
%\usepackage{times}
%\usepackage[T1]{fontenc}
% Or whatever. Note that the encoding and the font should match. If T1
% does not look nice, try deleting the line with the fontenc.

%\usepackage{fontspec}
%\setmonofont{CMU Typewriter Text}
%\setmonofont{Consolas}

%%%%%%%%%%%%%%%%%%%%%% start my preamble %%%%%%%%%%%%%%%%%%%%%%

\addtobeamertemplate{navigation symbols}{}{%
    \usebeamerfont{footline}%
    \usebeamercolor[fg]{footline}%
    \hspace{1em}%
    \insertframenumber/\inserttotalframenumber
}


\usepackage{graphicx}
\usepackage{amsmath, amssymb, amsthm}
\usepackage{bbm}
\usepackage{mathrsfs}
\usepackage{xcolor}
\usepackage{fancyvrb}

% Quotes at start of chapters / sections
\usepackage{epigraph}  
%\renewcommand{\epigraphflush}{flushleft}
%\renewcommand{\sourceflush}{flushleft}
\renewcommand{\epigraphwidth}{6in}

%% Fonts

%\usepackage[T1]{fontenc}
\usepackage{mathpazo}
%\usepackage{fontspec}
%\defaultfontfeatures{Ligatures=TeX}
%\setsansfont[Scale=MatchLowercase]{DejaVu Sans}
%\setmonofont[Scale=MatchLowercase]{DejaVu Sans Mono}
%\setmathfont{Asana Math}
%\setmainfont{Optima}
%\setmathrm{Optima}
%\setboldmathrm[BoldFont={Optima ExtraBlack}]{Optima Bold}

% Some colors

\definecolor{aquamarine}{RGB}{69,139,116}
\definecolor{midnightblue}{RGB}{25,25,112}
\definecolor{darkslategrey}{RGB}{47,79,79}
\definecolor{darkorange4}{RGB}{139,90,0}
\definecolor{dogerblue}{RGB}{24,116,205}
\definecolor{blue2}{RGB}{0,0,238}
\definecolor{bg}{rgb}{0.95,0.95,0.95}
\definecolor{DarkOrange1}{RGB}{255,127,0}
\definecolor{ForestGreen}{RGB}{34,139,34}
\definecolor{DarkRed}{RGB}{139, 0, 0}
\definecolor{DarkBlue}{RGB}{0, 0, 139}
\definecolor{Blue}{RGB}{0, 0, 255}
\definecolor{Brown}{RGB}{165,42,42}


\setlength{\parskip}{1.5ex plus0.5ex minus0.5ex}

%\renewcommand{\baselinestretch}{1.05}
%\setlength{\parskip}{1.5ex plus0.5ex minus0.5ex}
%\setlength{\parindent}{0pt}

% Typesetting code
\definecolor{bg}{rgb}{0.95,0.95,0.95}
\usepackage{minted}
\setminted{mathescape, frame=lines, framesep=3mm}
\usemintedstyle{friendly}
%\newminted{python}{}
%\newminted{c}{mathescape,frame=lines,framesep=4mm,bgcolor=bg}
%\newminted{java}{mathescape,frame=lines,framesep=4mm,bgcolor=bg}
%\newminted{julia}{mathescape,frame=lines,framesep=4mm,bgcolor=bg}
%\newminted{ipython}{mathescape,frame=lines,framesep=4mm,bgcolor=bg}


\newcommand{\Fact}{\textcolor{Brown}{\bf Fact. }}
\newcommand{\Facts}{\textcolor{Brown}{\bf Facts }}
\newcommand{\keya}{\textcolor{turquois4}{\bf Key Idea. }}
\newcommand{\Factnodot}{\textcolor{Brown}{\bf Fact }}
\newcommand{\Eg}{\textcolor{ForestGreen}{Example. }}
\newcommand{\Egs}{\textcolor{ForestGreen}{Examples. }}
\newcommand{\Ex}{{\bf Ex. }}



\renewcommand{\theFancyVerbLine}{\sffamily
    \textcolor[rgb]{0.5,0.5,1.0}{\scriptsize {\arabic{FancyVerbLine}}}}

\newcommand{\navy}[1]{\textcolor{Blue}{\bf #1}}
\newcommand{\brown}[1]{\textcolor{Brown}{\sf #1}}
\newcommand{\green}[1]{\textcolor{ForestGreen}{\sf #1}}
\newcommand{\blue}[1]{\textcolor{Blue}{\sf #1}}
\newcommand{\navymth}[1]{\textcolor{Blue}{#1}}
\newcommand{\emp}[1]{\textcolor{DarkOrange1}{\bf #1}}
\newcommand{\red}[1]{\textcolor{Red}{\bf #1}}

% Symbols, redefines, etc.

\newcommand{\code}[1]{\texttt{#1}}

\newcommand{\argmax}{\operatornamewithlimits{argmax}}
\newcommand{\argmin}{\operatornamewithlimits{argmin}}

\DeclareMathOperator{\cl}{cl}
\DeclareMathOperator{\interior}{int}
\DeclareMathOperator{\Prob}{Prob}
\DeclareMathOperator{\determinant}{det}
\DeclareMathOperator{\trace}{trace}
\DeclareMathOperator{\Span}{span}
\DeclareMathOperator{\rank}{rank}
\DeclareMathOperator{\cov}{cov}
\DeclareMathOperator{\corr}{corr}
\DeclareMathOperator{\var}{var}
\DeclareMathOperator{\mse}{mse}
\DeclareMathOperator{\se}{se}
\DeclareMathOperator{\row}{row}
\DeclareMathOperator{\col}{col}
\DeclareMathOperator{\range}{rng}
\DeclareMathOperator{\dimension}{dim}
\DeclareMathOperator{\bias}{bias}


% mics short cuts and symbols
\newcommand{\st}{\ensuremath{\ \mathrm{s.t.}\ }}
\newcommand{\setntn}[2]{ \{ #1 : #2 \} }
\newcommand{\cf}[1]{ \lstinline|#1| }
\newcommand{\fore}{\therefore \quad}
\newcommand{\tod}{\stackrel { d } {\to} }
\newcommand{\toprob}{\stackrel { p } {\to} }
\newcommand{\toms}{\stackrel { ms } {\to} }
\newcommand{\eqdist}{\stackrel {\textrm{ \scriptsize{d} }} {=} }
\newcommand{\iidsim}{\stackrel {\textrm{ {\sc iid }}} {\sim} }
\newcommand{\1}{\mathbbm 1}
\newcommand{\dee}{\,{\rm d}}
\newcommand{\given}{\, | \,}
\newcommand{\la}{\langle}
\newcommand{\ra}{\rangle}

\newcommand{\boldA}{\mathbf A}
\newcommand{\boldB}{\mathbf B}
\newcommand{\boldC}{\mathbf C}
\newcommand{\boldD}{\mathbf D}
\newcommand{\boldM}{\mathbf M}
\newcommand{\boldP}{\mathbf P}
\newcommand{\boldQ}{\mathbf Q}
\newcommand{\boldI}{\mathbf I}
\newcommand{\boldX}{\mathbf X}
\newcommand{\boldY}{\mathbf Y}
\newcommand{\boldZ}{\mathbf Z}

\newcommand{\bSigmaX}{ {\boldsymbol \Sigma_{\hboldbeta}} }
\newcommand{\hbSigmaX}{ \mathbf{\hat \Sigma_{\hboldbeta}} }

\newcommand{\RR}{\mathbbm R}
\newcommand{\NN}{\mathbbm N}
\newcommand{\PP}{\mathbbm P}
\newcommand{\EE}{\mathbbm E \,}
\newcommand{\XX}{\mathbbm X}
\newcommand{\ZZ}{\mathbbm Z}
\newcommand{\QQ}{\mathbbm Q}

\newcommand{\fF}{\mathcal F}
\newcommand{\dD}{\mathcal D}
\newcommand{\lL}{\mathcal L}
\newcommand{\gG}{\mathcal G}
\newcommand{\hH}{\mathcal H}
\newcommand{\nN}{\mathcal N}
\newcommand{\pP}{\mathcal P}




\title{Scientific Computing in Economics and Finance\\
    Past, Present and Future}

\author{John Stachurski}
\institute{Tokyo College and Australian National University}


\date{April 25th 2025}


\begin{document}

\begin{frame}
  \titlepage
\end{frame}





\section{Introduction}



\begin{frame}
    \frametitle{What do economists do?}

    \begin{itemize}
        \item Set interest rates at the BOJ
            \vspace{0.3em}
        \item Forecast tax revenues / assess policy changes
            \vspace{0.3em}
        \item Study competition policy (e.g., mergers, monopolies)
            \vspace{0.3em}
        \item Study pension plans
            \vspace{0.3em}
        \item Study the determinants of long run growth
            \vspace{0.3em}
        \item Study the impact of COVID / COVID relief policy
    \end{itemize}

            \vspace{0.3em}
            \vspace{0.3em}

    \Eg What will be the impact of a 100BPS $\uparrow$ in the federal funds
    rate on unemployment in one year?

\end{frame}


\begin{frame}
    \frametitle{Is economics a science?}

    \brown{The goal of the scientist is to comprehend the phenomena of the
    universe that he observes around him.}

            \vspace{0.3em}
    \brown{To prove that he understands he must be able to predict.}

            \vspace{0.3em}
    \brown{To predict quantitatively one must have a mechanism for producing
    numbers.}

            \vspace{0.3em}
    \brown{This necessarily entails a mathematical model.}

            \vspace{0.3em}
            \vspace{0.3em}
            \vspace{0.3em}
     -- Richard Bellman (1920 -- 1984)

\end{frame}

\begin{frame}
    \frametitle{So what is economic science?}

    Why is economics different to astrophysics, chemistry, etc.?

    \begin{itemize}
        \item Economic outcomes depend on human choices
            \vspace{0.3em}
        \item Human choices depend on beliefs, incentives, etc.
            \vspace{0.3em}
        \item Economic processes are nonstationary (no immutable laws)
    \end{itemize}

            \vspace{0.3em}
            \vspace{0.3em}
            \vspace{0.3em}

    \Eg The Lucas critique: it is naive to try to predict the effects
        of a change in economic policy entirely on the basis of relationships
        observed in historical data

\end{frame}

\begin{frame}
    \frametitle{Extended example: the cobweb model}

    An ``old'' economic model

    \begin{itemize}
        \item Benner (1876)
        \item Haas and Ezekil (1926)
        \item Ricci (1930)
        \item Kaldor (1934, 1938)
        \item Ezekil (1938)
        \item Rosen, Murphy, and Scheinkman (1994)
    \end{itemize}

\end{frame}

\begin{frame}
    
    \begin{figure}
        \centering
        \scalebox{.56}{\includegraphics{hog_prices.pdf}}
    \end{figure}

\end{frame}


\begin{frame}

    Ordinary models of supply and demand don't generate these cycles.

    \begin{itemize}
        \item find $p$ and $q$ from $q^d(p) = q^s(p)$
    \end{itemize}

    \begin{figure}
        \centering
        \scalebox{.36}{\includegraphics{hog_supply_demand.pdf}}
    \end{figure}
    

\end{frame}

\begin{frame}
    
    Hypotheses: 

    \begin{itemize}
        \item Farmers need time to raise hogs (say, one ``period'')
            \vspace{0.3em}
        \item Farmers forecast future prices using past and current prices
    \end{itemize}

    Outcomes:

    \begin{enumerate}
        \item Suppose price is currently high
            \vspace{0.3em}
        \item Farmers $\uparrow$ capacity, shift towards hog production
            \vspace{0.3em}
        \item Next period, high supply floods the market, prices $\downarrow$
            \vspace{0.3em}
        \item Seeing this low price, farmers $\downarrow$ capacity
            \vspace{0.3em}
        \item Next period, supply is low and prices $\uparrow$ ...
    \end{enumerate}

\end{frame}

\begin{frame}

    In this scenario,
    
    $$
        q^d(p_t) = q^s(p^e_{t-1})
    $$

    \begin{itemize}
        \item $p^e_{t-1}$ is the \textbf{expected} time $t$ price, formed at $t-1$
    \end{itemize}

    \vspace{1em}
    \vspace{1em}

    But how to farmers form expectations?

\end{frame}

\begin{frame}
    
    First guess:
    %
    \begin{equation*}
        p^e_{t-1} = p_{t-1}
    \end{equation*}

    So now we have

    $$
        q^d(p_t) = q^s(p_{t-1})
    $$

    Solving for $p_t$ gives

    $$
        p_t = f(p_{t-1})
        \quad \text{where} \quad
        f(p) = (q^d)^{-1} (q^s(p))
    $$

\end{frame}

\begin{frame}
    
    \begin{figure}
        \centering
        \scalebox{.56}{\includegraphics{hog_45.pdf}}
    \end{figure}

\end{frame}

\begin{frame}
    
    \begin{figure}
        \centering
        \scalebox{.56}{\includegraphics{hog_backward_ts.pdf}}
    \end{figure}

\end{frame}



\begin{frame}
    
    The model replicates cycles --- but there are problems!

    \vspace{1em}
    \vspace{1em}
    Predictions are \emp{very sensitive} to how we model \emp{expectations}

    \vspace{1em}
    \vspace{1em}
    \vspace{1em}
    \Eg Suppose we switch to $p_{t-1}^e = \alpha p_{t-1} + (1-\alpha) p^e_{t-2}$ 

    \vspace{1em}
    \vspace{1em}
    \begin{itemize}
        \item Called ``adaptive expectations''
    \end{itemize}

\end{frame}


\begin{frame}
    
    \begin{figure}
        \centering
        \scalebox{.42}{\includegraphics{hog_adaptive.pdf}}
    \end{figure}

\end{frame}


\begin{frame}

    Or we could use ``rational expectations''
    
    \begin{figure}
        \centering
        \scalebox{.36}{\includegraphics{hog_supply_demand.pdf}}
    \end{figure}

\end{frame}


\begin{frame}

    Lessons:
    %
    \begin{itemize}
        \item Model predictions are very sensitive to behavior, expectations
            \vspace{1em}
        \item Modeling human behavior is essential
            \vspace{1em}
        \item The ``right'' way to model humans is unclear
    \end{itemize}
    
        \vspace{1em}
        \vspace{1em}
    Summary: economic modeling is hard -- but we shouldn't give up!

\end{frame}


\begin{frame}
    \frametitle{Scientific computing in economics}

    \brown{Economists are relative newcomers to the
        field of computational sciences...}

            \vspace{0.3em}
    \brown{Economists have long been influenced by dogmatic tribalism...}

            \vspace{0.3em}
    \brown{It would appear that many (so called) `theories' have been poorly
    (if at all!) proven...}

            \vspace{0.3em}
    \brown{Computational models in economics are still often simplistic...}

    \vspace{0.3em}
    \vspace{0.3em}
    \vspace{0.3em}
    \begin{center}
        -- Consultant's report on HPC in economics
    \end{center}

\end{frame}


\begin{frame}
    
     \begin{tabular}{cl}  
         \begin{tabular}{c}
           \includegraphics[height=6cm]{moniac.jpg}
           \end{tabular}
           & \begin{tabular}{l}
             \parbox{0.6\linewidth}{%  change the parbox width as appropiate
                Actually economists are pioneers

                (William Phillips, 1949)
             }
         \end{tabular}  \\
    \end{tabular}
    

\end{frame}

\begin{frame}
    
    Examples of computational work in economics and finance

    - DSE
    - schelling
    - Pricing a European option 

\end{frame}


\begin{frame}
    \frametitle{Trends in scientific computing}

    compare languages, talk about jit compilers, illustrate with google jax.

\end{frame}


\begin{frame}
    \frametitle{The limits of computer power}

    * Linear assignment
    * Brute force optimization
        
\end{frame}

\begin{frame}
    
    * The importance of algorithms
    * How algorithms interact with advances in hardware/software

\end{frame}

\begin{frame}

    What about machine learning and AI?

    * Why machine learning is not enough -- current inflation unprecedented -- think about that word
    * Insufficient and nonstationary data -- forecasting GDP
    * The need for careful mathematical modeling
    
\end{frame}



\end{document}


